\chapter{Computation in Nature}

\begin{displayquote}
    \begin{center}
Nature is what all life is based on.
Nature encapsulates our reality.
Nature is within us and around us.
    \end{center}
\end{displayquote}

\noindent
The school of intuitionism in philosophy of mathematics, believes that mathematics solely is a product of human constructive intellect.
I however, suspect that mathematics is beyond anything man ever can or will comprehend, and that computation resides in nature as the foundation of our objective reality.
My beliefs are bound in both the known and provable aspects of nature such as potential and kinetic energy, which can be formalised and conceptualised using mathematical techniques; but also the unpredictable and uncovered side of nature.
Take for example the world of quantum and particle physics, which man constantly tries to describe with theories and mathematical models, just to be proven wrong when a unforeseen phenomenon breaks any hypothesis.
I see this as a piece of evidence that computation is something we as humans discover one fraction at a time, rather than something we have invented which can be used to scrutinize nature.

One of the most fundamental examples of computation in nature, is the presence of countability.
Within the following parenthesis I can count three dots ( $\cdots$ ); but how can I know they will always be counted as three dots?
If computation and mathematics was a product of human mental activities, who is to say the very way we perceive counting would not be subject to change.
Were it not for the fact that nature contains these mathematics, we would not be guaranteed a stable way of describing our universe.

Nature is not without computation when a particle can be in two states at the same time, we merely does not comprehend a large enough amount of the mathematics that lies in nature, to be able to model every part of it.
Otherwise quantum computing would be an elementary accomplishment\footnote{One could argue that producing quantum computers still would require some materialistic progress and significant amounts of money, but that is besides the point.}.

\bigskip
Given enough time, I also believe that humans will be able to expose more of the computation in nature, thus by evolution gaining a higher apprehension on the building blocks of our reality.
Allan Turing proposes the same learning be evolution, as a way to achieve artificial intelligence.
In \textit{Computing Machinery and Intelligence}, Turing proposes a \enquote{child machine} as the starting point for creating intelligence albeit artificial, but comparable to that of an adult.
Turing describes the \enquote{child machine} as a clean slate or \enquote{tabula rasa}, ready to gain knowledge and experiences.
As programmes develop this machine it will mutate, and their judgement of said mutations will work as a kind of natural selection.

This is identical how empiricism in epistemology view the of how we learn.
The evolutionary approach to learning is that of empiricism, as induction and sensorial input is used to build up an understanding of the world around us.
Moreover empiricism also oppose to the idea that we utilize innate ideas and deduction to infer instances in our reality by referencing general laws;
instead we infer principles and general laws by referring to our reality, just as I have argued previously.
Mathematics and computation resides in nature, ready for us to consume and discover.

\bigskip
To return to what I previously touched upon regarding the intuitionistic view of e.g. numbers, one could argue that mankind provides the syntax to computation while nature provides the semantics.
Ever since the dawn of human endeavours into the universe of mathematics humans have invented new syntax to be able to express newfound semantics.
To clarify, when we cannot convey some computation it is because we are missing the correct syntax to do so --- the semantic is already present in nature.

Because nature provides the semantics, any living being capable of inferring an appropriate syntax will be able to comprehend and utilize the potential of the computation in nature.
This correlates to the fact that any being, terrestrial or extra terrestrial, living in \enquote{our} perceived third dimensional world would arrive at the same conclusion as we have on subject such as: how does gravity work; or two stones and three stones gives five stones.

\bigskip
I have argued that mathematics computation exists in nature, but this spawns a need for an elaboration; does nature compute?
As in, can nature perform computations and perhaps conclude upon these.
Again I turn to evolution, this time let us focus on evolution as Charles Darwin described it in his book \textit{The Origin of Species} from 1909.
In particular let us investigate Darwin's finches --- about fifteen birds all belonging to different species but the same biological family.
These different finches had different beaks, depending on what type of food the would eat.
Evolution had throughout time changes the birds, to adapts to different life conditions.

I see this as nature doing a computation of whether or not a given evolutionary step is constructive or destructive.
The constructive changes are persisted in the evolutionary process, while nature performs what appears to be some sort of feedback loop.
Because of evolution I believe that nature is in deed capable of performing computations, and at a far more grandiose and complex scale than us humans.

\bigskip
To talk about nature without addressing computation or mathematics, is nearly impossible.
Nevertheless while some may argue that computation does not reside explicitly in nature, this is merely a question of definition.
Again, nature provides the semantics, so for humans to see explicit computation in nature, we must provide the syntax, thus mixing in our understanding of mathematics.
