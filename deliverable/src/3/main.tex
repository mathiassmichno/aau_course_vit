\chapter{Deliverable III}
\textcolor{gray}{%
\begin{itemize}
    \item Read the above mentioned paper How to Present a Paper in Theoretical Computer Science: A Speaker’s Guide for Students by Ian Parberry.
    \item Write a summary of the content of the paper (one paragraph).
    \item List five points mentioned in the paper where you think are your major weaknesses when giving a (slide) presentation and where you would like to improve in the future.
\end{itemize}}
\section{Summary of \textit{How to Present a Paper in Theoretical Computer Science: A Speaker’s Guide for Students} by Ian Parberry\footnote{https://www.csee.umbc.edu/csee/research/cra/etw98/speaker.pdf}}
The paper is sectioned into four topics; the first concerns the material to be presented and how to choose and organise it, the second is about the actual act of presenting, while the third section focusses on the aids one might use such as overhead and microphones, lastly the fourth section touched upon the post-presentation i.e.\ taking questions from the audience.
On how to choose and organise the material to be presented, Ian Parberry states that it is important to present key ideas and not the obvious.
He also underlines the importance of not going into to many details, as this will only help to loose the interest of the audience.
Furthermore he presents a structure consisting of; an introduction, a body, some technicalities, and lastly the conclusion.
This structure will ensure a Top-down approach, which will ease the general audience into the presentation, but also excite experts by presenting technicalities.
On how to perform the actual presentation, Parberry gives the sentence: \enquote{Tell them what you're going to tell them. Tell them. The tell them what you told them.}.
Coupled with repetition one should also try to remind the audience of any standard knowledge.
Moreover the author of the paper gives advice about dealing with anxiety, dressing appropriately and appear confident.
On using electronic aids when presenting, Ian Parberry advices the presenter to keep the content of slides to a minimum, i.e.\ not too much text, but key points, pictures and tables.
Parberry also speaks against an excessive number of slides and warns the presenter about pitfalls when using a microphone.
The question time after a presentation may be filled with people trying to gain reputation by tearing the presenter down, however try to remain composed, and do not be afraid to answer \enquote{I don't know}.
\section{My Possible Major Weaknesses}
\begin{enumerate}
    \item Not choosing the right material to be presented, i.e.\ including too much.
    \item Not using repetition properly and loosing the audience.
    \item Using information-free utterance and fashionable turns of phrase.
    \item Being anxious and loosing focus because of it.
    \item Being afraid of answering questions from the audience with \enquote{I don't know}.
\end{enumerate}
