\chapter{Deliverable II}
\textcolor{gray}{%
\begin{itemize}
        \item In groups, rewrite the title and abstract of your last year project. You should try to apply the rules/suggestions mentioned during the lecture and your target audience are computer science students that just finished their second year at the university. The basis for the abstract is your group work but it should be then individually improved (after the feedback you get) before it is added to your report.
        \item In the report write the names of all students participating in the preparation of the title/abstract.
\end{itemize}}
\section{Participants in Group Session}
\begin{multicols}{2}
    \begin{itemize}
        \item Claus W. Wiingreen, cwiing13
        \item Marc T. Thorgersen, mthorg13
        \item Mathias S. Michno, mmichn13
        \item Morten Pedersen, morped13
        \item Søren H. Frandsen, sfrand12
        \item Troels B. Krøgh, tkragh13
    \end{itemize}
\end{multicols}

\section{Title}
\begin{center}
    \textit{Old title:} \\
    \textbf{Timely Wireless Arduino Communication}\\
    \textit{New title:}\\
    \textbf{Distributed Communications Protocol for Wireless Embedded Device Networks}
\end{center}
\section{Abstract}
\begin{center}
    \textit{Old abstract:}
\end{center}
In this paper we explore the possibilities of designing and implementing a wireless single frequency communication protocol for embedded devices using cheap RF-modules.
We design simple protocol and further expand it to account for scenarios which can arise in a real-world setting.
We test the hardware in regard to reliability; the protocol is designed in regard to a naive scenario and iterated upon to account for more realistic scenarios.
Each scenario is implemented as a model in UPPALL to verify the validity of the design and to calculate probabilities of successful networks being established.
We implement the design onto Arduino devices with attached RF-modules.
The protocol is not specified to any distinct use case and allows for user-code to be run alongside the protocol as well as modifiable payloads to allow for different use cases such as a fire alarm network or home automation. 
\begin{center}
    \textit{New abstract:}
\end{center}
When trying to communicate wirelessly over a single frequency, many problems and obstacles must be overcome in order to establish a reliable distributed network. 
We propose a graph based design for a distributed wireless communication protocol using single frequency Radio Frequency-modules based on the excising concept of Time Division Multiple Access. 
We model and verify the proposed design of the protocol, using the model-checker tool UPPAAL, as a part of a model-driven development process. 
Furthermore we solve oncoming problems such as simultaneous device activation and packet-loss, using the model-driven approach with UPPAAL as the main verification tool. 
Lastly we implement the verified protocol design on embedded systems configured with single frequency Radio Frequency-modules, such that arbitrary data can be transmitted. 
We design and implement the protocol with the intent of reuse, such that any application can disregard the issues coupled with creating and maintaining a distributed wireless network using a single frequency.
